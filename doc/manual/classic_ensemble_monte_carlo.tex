
% ==============================================================================
\chapter{Classic Ensemble Monte Carlo}   \label{emc}
% ==============================================================================

\section{Introduction}
The Classic Ensemble Monte Carlo (CEMC) tool is designed for simulating two-dimensional MOSFET devices by means of solving the Boltzmann transport
equation using the Ensemble Monte Carlo algorithm. It is based on Fortran code developed
by Prof. Dragica Vasileska from Arizona State University. The original Fortran code has been translated to C and optimized.

Currently the implemented scattering mechanism include Coulomb scattering (Brooks-Herring approach),
acoustic phonon scattering and intervalley scattering (zero-order f and g absorption and emission processes).
The implemented 2D Monte Carlo Poisson solver solves the 2D Poisson-equation using the successive over-relaxation (SOR) method.

%% ==============================================================================
%% ==============================================================================
%\section{Building Instructions} \label{boltzmann:building}
%% ==============================================================================
%% ==============================================================================

%The CEMC simulator is generated by the central ViennaWD CMake build system.

%\begin{lstlisting}
%$> cd ViennaWD/
%$>
%\end{lstlisting}

%The Makefile compiles the source files with the command

%\begin{lstlisting}
%$> gcc -c -Wall -O3 -g3 -pedantic
%\end{lstlisting}

%and creates an executable \texttt{viennaemc} which can be simply executed by typing

%\begin{lstlisting}
%$> ./viennaemc
%\end{lstlisting}

%\TIP{The input files which need to be in the same directory as the executable are read automatically.}

%To remove the executable and object files type
%\begin{lstlisting}
%$> make clean
%\end{lstlisting}

% ==============================================================================
% ==============================================================================
\section{Building Information} %\addcontentsline{toc}{chapter}{Building Instructions} \label{building}
% ==============================================================================
% ==============================================================================

% ==============================================================================
\subsection{Dependencies} \label{boltzmann:dependencies}
% ==============================================================================
%The CEMC tool depends on the math library (\texttt{libm}), which is available on all Unix systems and automatically
%discovered and linked by the CMake build system.

In the following the dependencies of the CEMC simulator are presented.

\begin{itemize}
  \item C-Compiler (e.g. GCC~\cite{gcc} Version $\geq4.4$)
  \item Math library (Unix Standard)
\end{itemize}

%The ViennaEMC library uses only standard \textit{C} libraries and does not depend on any external libraries.
%Only a suitable C compiler (e.g. GCC) is necessary.


% ==============================================================================
\section{Examples} \label{boltzmann:examples}
% ==============================================================================
Three examples are shipped with this package.

The first example can be found in the directory

\texttt{examples/example01}

and contains input files with default values as described in Section~\ref{boltzmann:sim},
Table~\ref{tab:simpar} and Table~\ref{tab:devdim} (MOSFET device in on-state).

The second example is located in the directory

\texttt{examples/example02}

and contains the same device geometry with zero drain voltage (\texttt{inputVd = 0.0}) representing a MOSFET device in off-state.


The first two examples can be executed by the following.
\begin{lstlisting}
$> cd ViennaWD/build/classic_ensemble_monte_carlo/
$> ./emc ../../classic_ensemble_monte_carlo/examples/example0x/device_dimensions.in
     ../../classic_ensemble_monte_carlo/examples/example0x/simulation_parameters.in
\end{lstlisting}

The third example is located in the directory

\texttt{examples/example03}

and contains an example of how to use the simulator kernel in an external C++ program. The example consists of a \texttt{main.cpp} C++ source code file where a minimalist code required to run the simulator is given.
The following data structures
\begin{itemize}
\item \texttt{const\_t constants;}
\item \texttt{geometry\_t *geometry;}
\item \texttt{scatpar\_t *scatpar;}
\item \texttt{phys\_quant\_t *phys\_quantities;}
\item \texttt{output\_t *outputdata;}
\end{itemize}
must be declared and allocated as shown. Then the user needs to initialize the simulation parameters shown in table \ref{tab:simpar} as well as the device dimension parameters shown in table \ref{tab:devdim}. Also the following initialization functions
\begin{itemize}
\item \texttt{oooMatParInitialization(\&constants);}
\item \texttt{oooScatParInitialization(scatpar);}
\item \texttt{oooDeviceStructureInitialization(constants, geometry, scatpar);}
\end{itemize}
must be called as shown to initialize the data structures. \\
The simulator itself is started by calling: \\
\texttt{outputdata = EMC(constants, geometry, scatpar, phys\_quantities);} \\
This function has a return value of type \texttt{output\_t *} and returns a pointer to the output data which can be processed in the next step. The output data structure is described in section \ref{boltzmann:out}. Finally the allocated memory needs to be freed in order to avoid memory leaks before exiting the program.

%\TIP{Keep the default parameters described in Section \ref{boltzmann:sim}, tables \ref{tab:mc2d}, \ref{tab:matpar}, \ref{tab:scatflag} and \ref{tab:scatpar} when compiling an executable for running the examples. Alternatively all source files and the Makefile can be copied to the corresponding example folder and compiled there using \texttt{make}.}

% ==============================================================================
% ==============================================================================
\section{Simulation Control} \label{boltzmann:sim}
% ==============================================================================
% ==============================================================================

The simulations can be controlled by two input files:
\begin{itemize}
\item \texttt{simulation\_parameters.in}
\item \texttt{device\_dimensions.in},
\end{itemize}
which are read in the main program as well as constants defined in \texttt{emc.h} and silicon material parameters implemented in \texttt{mat\_par\_initialization.c}
located in the \texttt{src} folder.
This section describes all relevant parameters.

\subsection{Input file format}

Simulation parameters contained in \texttt{simulation\_parameters.in} are described in table \ref{tab:simpar}.

\begin{table}[ht!]
\centering
\begin{tabular}{|l|p{7cm}|c|c|}
\hline
\textbf{Variable}                    &  \textbf{Definition}                           &  \textbf{Default} \\
\hline
\texttt{scatpar->dt}                 &  differential time step in $[s]$               &  $1.5e^{-16}$ \\
\hline
\texttt{scatpar->totalTime}          &  total time to be simulated in $[s]$           &  $10e^{-12}$ \\
\hline
\texttt{scatpar->transientTime}      &  duration of the initial transient in $[s]$    &  $3.0e^{-12}$ \\
\hline
\texttt{scatpar->averTime}           &  averaging time in $[s]$                       &  $2.0e^{-13}$ \\
\hline
\texttt{phys\_quantities->inputVs}   &  input source voltage in $[V]$                 &  $0.0$ \\
\hline
\texttt{phys\_quantities->inputVd}   &  input drain voltage in $[V]$                  &  $1.0$\\
\hline
\texttt{phys\_quantities->inputVg}   &  input gate voltage in $[V]$                   &  $1.0$ \\
\hline
\texttt{phys\_quantities->inputVsub} &  input substrate voltage in $[V]$              &  $0.0$\\
\hline
\texttt{constants.omega}             &  relaxation factor for the SOR Poisson solver, should be in the range $1 \leq \omega \leq 2$  &  $1.8$ \\
\hline
\texttt{constants.tolerance}         &  convergence tolerance for the Poisson solver  &  $1e^{-4}$\\
\hline
\end{tabular}
\caption{The input variables in \texttt{simulaton\_parameters.in} are described.}
\label{tab:simpar}
\end{table}

The device parameters contained in \texttt{device\_dimensions.in} are described in table \ref{tab:devdim}.

\begin{table}[ht!]
\centering
\begin{tabular}{|l|p{8.0cm}|c|c|}
\hline
\textbf{Variable}                 &  \textbf{Definition}                              &  \textbf{Default} \\
\hline
\texttt{geometry->lengthSD}       &  length of the source and drain contacts in $[m]$ &  $50e^{-9}$ \\
\hline
\texttt{geometry->depthSD}        &  depth of the source and drain contacts in $[m]$  &  $30e^{-9}$ \\
\hline
\texttt{geometry->lengthG}        &  length of the gate in $[m]$                      &  $25e^{-9}$ \\
\hline
\texttt{geometry->depthB}         &  bulk depth in $[m]$                              &  $70e^{-9}$ \\
\hline
\texttt{geometry->deviceWidth}    &  width of the device in $[m]$                     &  $1e^{-6}$ \\
\hline
\texttt{geometry->oxideThickness} &  thickness of the $SiO_2$ layer in $[m]$          &  $1.2e^{-9}$\\
\hline
\texttt{geometry->meshSize}       &  mesh size in $[m]$                               &  $1e^{-9}$ \\
\hline
\texttt{scatpar->dopingCon{[0]}}  &  doping concentration for region 1 in $[m^{-3}]$  &  $5e^{25}$\\
\hline
\texttt{scatpar->dopingCon{[1]}}  &  doping concentration for region 2 in $[m^{-3}]$  &  $5e^{25}$ \\
\hline
\texttt{scatpar->dopingCon{[2]}}  &  doping concentration for region 3 in $[m^{-3}]$  &  $-5e^{24}$\\
\hline
\texttt{scatpar->dopingCon{[3]}}  &  doping concentration for region 4 in $[m^{-3}]$  &  $-5e^{23}$\\
\hline
\end{tabular}
\caption{The input variables in \texttt{device\_dimensions.in} are described.}
\label{tab:devdim}
\end{table}



\subsection{\texttt{emc.h} header file description}

The \texttt{emc.h} header file contains constant definitions as well as data structure and function declarations.
Relevant macros are depicted in Table~\ref{tab:emc}.

\clearpage

\begin{table}[ht!]
\centering
\begin{tabular}{|l|p{9cm}|c|c|}
\hline
\textbf{Constant}        & \textbf{Definition}                               & \textbf{Default} \\
\hline
\texttt{\#define MAXNX}  & maximum mesh x coordinate                         &  $200$ \\
\hline
\texttt{\#define MAXNY}  & maximum mesh y coordinate                         &  $200$ \\
\hline
\texttt{\#define DOPREG} & maximum number of doping regions                  &  $4$ \\
\hline
\texttt{\#define MAXSC}  & maximum number of scattering mechanisms           &  $10$ \\
\hline
\texttt{\#define NLEV}   &  number of energy levels in the scattering table  &  $1000$ \\
\hline
\texttt{\#define MAXEN}  &  maximum number of electrons                      &  $1000000$\\
\hline
\end{tabular}
\caption{The constant definitions in \texttt{emc.h} are shown.}
\label{tab:emc}
\end{table}

\TIP{The user may need to modify the constants \texttt{MAXNX} and \texttt{MAXNY} if the default values exceed the ratios $\frac{device \: x-length}{mesh \: size}$ and $\frac{device \: y-length}{mesh \: size}$.}

\TIP{\texttt{MAXEN} may also be modified by the user in case the simulator reports the error \textit{Actual number of electrons exceeds \texttt{MAXEN}}.}


\subsection{Material \& Scattering parameters}

The material parameters for Silicon are defined in \texttt{mat\_par\_initialization.c}. The table \ref{tab:matpar} shows material parameter variables that may be modified by the user if needed.

\begin{table}[ht!]
\centering
\begin{tabular}{|l|l|c|}
\hline
\textbf{Variable}             & \textbf{Definition}                       & \textbf{Default value} \\
\hline
\texttt{TL}                   & lattice temperature in $[K]$              & $300.0$ \\
\hline
\texttt{am\_l}                & lateral effective mass                    & $0.91$ \\
\hline
\texttt{am\_t}                & transversal effective mass                & $0.19$ \\
\hline
\texttt{constpar->eps\_sc}    & relative permittivity of Silicon          & $11.8 \: \varepsilon_0$\\
\hline
\texttt{constpar->eps\_ox}    & relative permittivity of $SiO_{2}$        & $3.9 \: \varepsilon_0$\\
\hline
\texttt{constpar->Ni}         & intrinsic carrier density in Silicon      & $1.45e^{16}$ \\
\hline
\texttt{constpar->delta\_Ec}  & half band gap energy normalized by $V_T$  & $0.575 / \texttt{constpar->vt}$ \\
\hline
\texttt{constpar->density}    & Silicon density in $[kg/m^{3}]$           & $2329.0$ \\
\hline
\texttt{scatpar->vsound}      & sound velocity in Silicon in $[m/s]$      & $9040.0$ \\
\hline
\texttt{constpar->af}         & non-parabolicity factor                   & $0.5$ \\
\hline
\end{tabular}
\caption{User modifiable material parameters and their respective default values in \texttt{mat\_par\_initialization.c} are shown.}
\label{tab:matpar}
\end{table}

Scattering mechanisms are controlled by setting the variable switches listed in table \ref{tab:scatflag}.

\TIP{By default, coulomb scattering, acoustic phonon scattering and intervalley zero-order scattering are enabled whereas intervalley first-order scattering and surface roughness are disabled.}

\begin{table}[ht!]
\centering
\begin{tabular}{|l|p{5cm}|c|}
\hline
\textbf{Variable}                    & \textbf{Mechanism}                                  & \textbf{Value} \\
\hline
\texttt{scatpar->coulombscattering}  & coulomb scattering                                  & 0 .. off, 1 .. on \\
\hline
\texttt{scatpar->acousticscattering} & acoustic phonon scattering                          & 0 .. off, 1 .. on \\
\hline
\texttt{scatpar->intervalley0g}      & intervalley phonon scattering zero order g-process  & 0 .. off, 1 .. on \\
\hline
\texttt{scatpar->intervalley0f}      & intervalley phonon scattering zero order f-process  & 0 .. off, 1 .. on \\
\hline
\texttt{scatpar->intervalley1g}      & intervalley phonon scattering first order g-process & 0 .. off, 1 .. on \\
\hline
\texttt{scatpar->intervalley1f}      & intervalley phonon scattering first order f-process & 0 .. off, 1 .. on \\
\hline
\texttt{scatpar->surfaceroughness}   & surface roughness                                   & 0 .. off, 1 .. on \\
\hline
\end{tabular}
\caption{Scattering mechanism selection variables \texttt{mat\_par\_initialization.c} are shown.}
\label{tab:scatflag}
\end{table}

Parameters for the scattering processes can be modified by setting the corresponding variables listed in table \ref{tab:scatpar}.

\begin{table}[ht!]
\centering
\begin{tabular}{|l|p{8cm}|c|}
\hline
\textbf{Variable}          & \textbf{Mechanism}                                      & \textbf{Value} \\
\hline
\texttt{scatpar->sigma}    & acoustic deformation potential in $[eV]$                & $6.55$ \\
\hline
\texttt{scatpar->defpot0g} & zero-order g-process deformation potential in $[eV/m]$  & $5.23e^{10}$ \\
\hline
\texttt{scatpar->defpot0f} & zero-order f-process deformation potential in $[eV/m]$  & $5.23e^{10}$ \\
\hline
\texttt{scatpar->defpot1g} & 1st order g-process deformation potential in $[eV/m]$   & $0.0$ \\
\hline
\texttt{scatpar->defpot1f} & 1st order f-process deformation potential in $[eV/m]$   & $0.0$ \\
\hline
\texttt{scatpar->phonon0g} & zero-order g-process phonon energy in $[eV]$            & $63.0e^{-3}$ \\
\hline
\texttt{scatpar->phonon0f} & zero-order f-process phonon energy in $[eV]$            & $59.0e^{-3}$ \\
\hline
\texttt{scatpar->phonon1g} & 1st order g-process phonon energy in $[eV]$             & $27.8e^{-3}$ \\
\hline
\texttt{scatpar->phonon1f} & 1st order f-process phonon energy in $[eV]$             & $29.0e^{-3}$ \\
\hline
\end{tabular}
\caption{Scattering parameters with their default values in \texttt{mat\_par\_initialization.c} are shown.}
\label{tab:scatpar}
\end{table}


% ==============================================================================
% ==============================================================================
\section{Simulation output} \label{boltzmann:out}
% ==============================================================================
% ==============================================================================

\subsection{Output data structure}

All output quantities are saved in the data structure \texttt{output\_t} defined in \texttt{emc.h}. This data structure can be accessed in the \texttt{main(...)} function directly or it can be written into files by calling \texttt{oooWriteOutput(...)} which generates most of the output files described in the next section. The structure itself is shown in the code snipped below.

\begin{verbatim}
typedef struct
{
  int totaltime;
  int x_max, y_max;
  int iterTotal;
  double x_axis[MAXNX],
         y_axis[MAXNY];

  currents_t *current_cumulative;
  curr_from_charge_t *current_from_charge;

  /* 2D quantities */
  double potential[MAXNX][MAXNY];
  double electrondensity[MAXNX][MAXNY];

  double fieldXY_x[MAXNX][MAXNY],
         fieldXY_y[MAXNX][MAXNY];

  double currentdensityX[MAXNX][MAXNY],
         currentdensityY[MAXNX][MAXNY],
         currentdensity[MAXNX][MAXNY];

  /* quantities along x direction with y = 0 */
  double fieldX[MAXNX], fieldX2[MAXNX],
         fieldY[MAXNX], fieldY2[MAXNY];
  double density[MAXNX];
  double sheetdensity[MAXNX];

  double velocityX[MAXNX],
         velocityY[MAXNX];
  double energy[MAXNX];
} output_t;
\end{verbatim}

It contains 1D and 2D data arrays which can be accessed by \texttt{for}-loops going from 0 to \texttt{x\_max} and \texttt{y\_max} respectively where \texttt{x\_max} and \texttt{y\_max} are the maximum mesh dimensions. \\
\texttt{int totaltime} contains the total simulation runtime in seconds. \\
\texttt{int iterTotal} contains the total number of iterations used to access the pointers \\
\texttt{*current\_cumulative} and \texttt{*current\_from\_charge} holding source and drain current data. Those pointers point to an array of size iterTotal of items of the type

\begin{verbatim}
typedef struct
{
  double Is_cumul, Id_cumul;
  double Is_momentary, Id_momentary;
} currents_t;
\end{verbatim}
and \clearpage
\begin{verbatim}
typedef struct
{
  double Time;
  double sourceCurr, drainCurr;
  double sourceFactor, drainFactor;
} curr_from_charge_t;
\end{verbatim}
respectively. \\
The type \texttt{currents\_t} contain cumulative source (\texttt{Is\_cumul}) and drain (\texttt{Id\_cumul}) currents as well as momentary source (\texttt{Is\_momentary}) and drain (\texttt{Id\_momentary}) currents for each iteration. \\
The type \texttt{curr\_from\_charge\_t} contains the simulation time (\texttt{Time}) at each iteration, the source (\texttt{sourceCurr}) and drain (\texttt{drainCurr}) currents calculated from charge and the source and drain factors which represent the no. of particles leaving the given contact - no. of eliminated particles + no. of created particles.

\subsection{Output file format}

The simulator generates the following output files: \\

\begin{lstlisting}
cur_from_charge_SD.csv
current_cumulative.csv
current_densityX.csv
current_densityY.csv
current_density.csv
electron_density.csv
fields_density_x.csv
fieldXY.csv
potential.csv
rateAcoustic.csv
rateCoulomb.csv
rateIntervalleyAbf.csv
rateIntervalleyAbg.csv
rateIntervalleyEmf.csv
rateIntervalleyEmg.csv
sheet_density_x.csv
total_simulation_time.csv
v_e_aver.csv
x_axis.csv
y_axis.csv
\end{lstlisting}

Tables \ref{tab:fileformat} and \ref{tab:fileformat2} show the file format for the output files, which is based on the
comma-separated values (CSV) format. Visualization aspects are discussed in Appendix A.
%For plotting purposes MATLAB~\cite{matlab}, the ParaView~\cite{paraview} visualization tool as well as Gnuplot~\cite{gnuplot} can be used.

\begin{table}[ht!]
\centering
%\begin{tabular}{|c|c|c|c|c|}
\begin{tabular}{|m{2.7cm}|m{2.7cm}|m{2.7cm}|m{2.7cm}|m{2.7cm}|}
\hline
  \multicolumn{5}{|l|}{\textbf{FILE: \texttt{cur\_from\_charge\_SD.csv}}} \\
\hline
  column 1 & column 2 & column 3 & column 4 & column 5  \\
\hline
  time $[ps]$ & source current $[A]$ & drain current $[A]$ & source factor & drain factor \\
\hline \hline
  \multicolumn{5}{|l|}{\textbf{FILE: \texttt{currentcum.csv}}} \\
\hline
  column 1 & column 2 & \multicolumn{3}{|l|}{column 3} \\
\hline
  time $[ps]$ & cumulative source current $[mA]$ & cumulative drain current $[mA]$ & momentary source current $[mA]$ & momentary drain current $[mA]$ \\
\hline \hline
  \multicolumn{5}{|l|}{\textbf{FILE: \texttt{current\_densityX.csv}}} \\
\hline
  column 1 & column 2 & \multicolumn{3}{|l|}{column 3} \\
\hline
  x-coordinate & y-coordinate & \multicolumn{3}{|l|}{current density in x-direction $[mA/m^3]$} \\
\hline \hline
  \multicolumn{5}{|l|}{\textbf{FILE: \texttt{current\_densityY.csv}}} \\
\hline
  column 1 & column 2 & \multicolumn{3}{|l|}{column 3} \\
\hline
  x-coordinate & y-coordinate & \multicolumn{3}{|l|}{current density in y-direction $[mA/m^3]$} \\
\hline \hline
  \multicolumn{5}{|l|}{\textbf{FILE: \texttt{current\_density.csv}}} \\
\hline
  column 1 & column 2 & \multicolumn{3}{|l|}{column 3} \\
\hline
  x-coordinate & y-coordinate & \multicolumn{3}{|l|}{current density $[mA/m^3]$} \\
\hline \hline
  \multicolumn{5}{|l|}{\textbf{FILE: \texttt{electron\_density.csv}}} \\
\hline
  column 1 & column 2 & \multicolumn{3}{|l|}{column 3} \\
\hline
  x-coordinate & y-coordinate & \multicolumn{3}{|l|}{electron density $[1/m^3]$} \\
\hline \hline
  \multicolumn{5}{|l|}{\textbf{FILE: \texttt{fields\_density\_x.csv}}} \\
\hline
  column 1 & column 2 & \multicolumn{3}{|l|}{column 3}  \\
\hline
  electric field x-component $[V/m]$ & electric field y-component $[V/m]$ & \multicolumn{3}{|l|}{averaged sheet density $[1/m^2]$} \\
\hline \hline
  \multicolumn{5}{|l|}{\textbf{FILE: \texttt{fieldXY.csv}}} \\
\hline
  column 1 & column 2 & column 3 & \multicolumn{2}{|l|}{column 4} \\
\hline
  x-coordinate & y-coordinate & electric field x-component in $[V/m]$ & \multicolumn{2}{|l|}{electric field y-component in $[V/m]$} \\
\hline \hline
  \multicolumn{5}{|l|}{\textbf{FILE: \texttt{potential.csv}}} \\
\hline
  column 1 & column 2 & \multicolumn{3}{|l|}{column 3}  \\
\hline
  x-coordinate & y-coordinate & \multicolumn{3}{|l|}{potential $[V]$} \\
\hline
\end{tabular}
\caption{File format for output files 1-8 is shown.}
\label{tab:fileformat}
\end{table}


\begin{table}[ht!]
\centering
%\begin{tabular}{|c|c|c|c|c|}
\begin{tabular}{|m{3.5cm}|m{3.5cm}|m{3.5cm}|m{3.5cm}|m{3.5cm}|}
\hline
  \multicolumn{5}{|l|}{\textbf{FILE: \texttt{rateAcoustic.csv}}} \\
\hline
  column 1 & \multicolumn{4}{|l|}{column 2}  \\
\hline \hline
  energy $[eV]$ & \multicolumn{4}{|l|}{acoustic scattering rate $[1/s]$} \\
\hline
  \multicolumn{5}{|l|}{\textbf{FILE: \texttt{rateCoulomb.csv}}} \\
\hline
  column 1 & \multicolumn{4}{|l|}{column 2}  \\
\hline
  energy $[eV]$ &  \multicolumn{4}{|l|}{coulomb scattering rate $[1/s]$} \\
\hline
  \multicolumn{5}{|l|}{\textbf{FILE: \texttt{rateIntervalleyAbf.csv}}} \\
\hline
  column 1 & \multicolumn{4}{|l|}{column 2}  \\
\hline
  energy $[eV]$ & \multicolumn{4}{|l|}{intervalley f-process absorption scattering rate $[1/s]$}  \\
\hline \hline
  \multicolumn{5}{|l|}{\textbf{FILE: \texttt{rateIntervalleyAbg.csv}}} \\
\hline
  column 1 & \multicolumn{4}{|l|}{column 2}  \\
\hline
  energy $[eV]$ & \multicolumn{4}{|l|}{intervalley g-process absorption scattering rate $[1/s]$} \\
\hline \hline
  \multicolumn{5}{|l|}{\textbf{FILE: \texttt{rateIntervalleyEmf.csv}}} \\
\hline
  column 1 & \multicolumn{4}{|l|}{column 2}  \\
\hline
  energy $[eV]$ & \multicolumn{4}{|l|}{intervalley f-process emission scattering rate $[1/s]$} \\
\hline \hline
  \multicolumn{5}{|l|}{\textbf{FILE: \texttt{rateIntervalleyEmg.csv}}} \\
\hline
  column 1 & \multicolumn{4}{|l|}{column 2}  \\
\hline
  energy $[eV]$ & \multicolumn{4}{|l|}{intervalley g-process emission scattering rate $[1/s]$} \\
\hline \hline
  \multicolumn{5}{|l|}{\textbf{FILE: \texttt{sheet\_density\_x.csv}}} \\
\hline
  \multicolumn{5}{|l|}{column 1}  \\
\hline
  \multicolumn{5}{|l|}{sheet density in $[1/m^2]$} \\
\hline \hline
  \multicolumn{5}{|l|}{\textbf{FILE: \texttt{total\_simulation\_time.csv}}} \\
\hline
  \multicolumn{5}{|l|}{column 1}  \\
\hline
  \multicolumn{5}{|l|}{total simulation runtime in $[s]$} \\
\hline \hline
  \multicolumn{5}{|l|}{\textbf{FILE: \texttt{v\_e\_aver.csv}}} \\
\hline
  column 1 & column 2 & \multicolumn{3}{|l|}{column 3}  \\
\hline
  mean x-velocity $[m/s]$ & mean y-velocity $[m/s]$ & \multicolumn{3}{|l|}{mean particle energy $[eV]$} \\
\hline \hline
  \multicolumn{5}{|l|}{\textbf{FILE: \texttt{x\_axis.csv}}} \\
\hline
  \multicolumn{5}{|l|}{column 1}  \\
\hline
  \multicolumn{5}{|l|}{x-axis mesh point coordinates in $[m]$} \\
\hline \hline
  \multicolumn{5}{|l|}{\textbf{FILE: \texttt{y\_axis.csv}}} \\
\hline
  \multicolumn{5}{|l|}{column 1}  \\
\hline
  \multicolumn{5}{|l|}{y-axis mesh point coordinates in $[m]$} \\
\hline
\end{tabular}
\caption{File format for output files 9-17 is shown.}
\label{tab:fileformat2}
\end{table}


\clearpage

% ==============================================================================
\section{License}
% ==============================================================================


Copyright (c) 2013, Institute for Microelectronics, TU Wien

Permission is hereby granted, free of charge, to any person obtaining a copy of this software and associated documentation files (the ”Software”), to deal in the Software without
restriction, including without limitation the rights to use, copy, modify, merge, publish, distribute, sublicense, and/or sell copies of the Software, and to permit persons to whom the
Software is furnished to do so, subject to the following conditions:

The above copyright notice and this permission notice shall be included in all copies or
substantial portions of the Software.

THE SOFTWARE IS PROVIDED ”AS IS”, WITHOUT WARRANTY OF ANY KIND, EXPRESS OR IMPLIED, INCLUDING BUT NOT LIMITED TO THE WARRANTIES OF
MERCHANTABILITY, FITNESS FOR A PARTICULAR PURPOSE AND NONINFRINGEMENT. IN NO EVENT SHALL THE AUTHORS OR COPYRIGHT HOLDERS BE LIABLE
FOR ANY CLAIM, DAMAGES OR OTHER LIABILITY, WHETHER IN AN ACTION OF
CONTRACT, TORT OR OTHERWISE, ARISING FROM, OUT OF OR IN CONNECTION
WITH THE SOFTWARE OR THE USE OR OTHER DEALINGS IN THE SOFTWARE.
